\documentclass[a4paper,10pt]{article}
\usepackage[utf8]{inputenc}
\usepackage{todonotes}
\usepackage{amsmath,amssymb}
\usepackage{url}
\usepackage{listings}
\usepackage{color}

%opening
\title{Solving Integer Optimization Problems with ZIMPL and SCIP}
\author{Gregor Hendel\and
        Thorsten Koch\and
        Jakob Witzig}

\definecolor{seagreen}{rgb}{0.18,0.74,0.56}
\definecolor{darkgreen}{rgb}{0.0,0.35,0.00}
\definecolor{navyblue}{rgb}{0.0,0.0,0.5}
\definecolor{steelblue}{rgb}{0.27,0.51,0.71}
\definecolor{siennabrown}{rgb}{0.63,0.32,0.18}
\definecolor{firebrickred}{rgb}{0.69,0.13,0.13}
\definecolor{gray75}{rgb}{0.75,0.75,0.75}

\lstdefinelanguage{zimpl}{%
   keywords={set,var,param,minimize,maximize,subto},%
   ndkeywords={read,as,comment,binary,integer,real,sum,forall,do,in,proj,vif,vabs,and,or,then,else,end},%
   sensitive,%
   showstringspaces=false,%
   morecomment=[l]\#,
   morestring=[b]",%
   keywordstyle=\color{red},%
   ndkeywordstyle=\color{navyblue},%
   commentstyle=\color{darkgreen},%
   stringstyle=\color{steelblue}%
}[keywords,comments,strings]%

\begin{document}

\maketitle

\section{The SCIP Optimization Suite}

\todo[inline]{Jakob}

\section{The Facility Location Problem}

\todo[inline]{Jakob}

\begin{align}
%
\begin{aligned}
& & \min \sum\limits_{j \in J} f_{j} y_{j} + \sum\limits_{i\in I, j\in J} c_{j,i} x_{j,i}\\
&\text{s.t.}& \sum\limits_{j \in J} : x_{j,i} &= 1 & \text{for all $i \in I$}\\
&& \sum\limits_{i \in I} d_{i} x_{j,i} & \leq s_{j} y_{j} & \text{for all $j \in J$}\\
&& (x,y) & \in \{0,1\}^{|J \times I|} \times \{0,1\}^{|J|}
\end{aligned}
\tag{SSCFLP}\label{eq:model}
\end{align}


\section{Building a ZIMPL Model}

The first part of this tutorial describes the translation of the mathematical model \eqref{eq:model} from the previous section into a file that SCIP can read and solve.
%
For this translation, we use the modeling language ZIMPL (Zuse Institute Mathematical Programming Language) which is part of the SCIP Optimization Suite.
%
As data source, we use a publicly available data set~\footnote{\url{http://or-brescia.unibs.it/instances/instances_sscflp}, ``Data set OR-library''} used by~\cite{Guastaroba14},
which contains a total of 36 different SSCFLP instances of varying sizes, ranging between 16--100 facilities and 50--1000 customers, respectively.
%
In order to highlight how ZIMPL reads data, we modified the original data format without changing the data itself.
%
The parameters of each instance are described in a single data file (extension \texttt{.dat}), which now has the following format:

\begin{itemize}
  \item the first line contains the number $|J|$ of facilities and $|I|$ of customers
  \item the next $|J|$ lines contain, for each facility $j \in J$, the index $j$ itself, the cost coefficient $f_{j}$ to open $j$ and its supply $s_{j}$.
  \item the next $|I|$ lines list the demand of each customer $i \in I$, one per line.
  \item the remaining $|I| \times |J|$ lines list the costs $c_{j,i}$ to link facility $j$ to customer $i$.
\end{itemize}

The task is to write a ZIMPL file that combines the data source and the mathematical model into an instance of SSCFLP.
%
For this purpose, we open a file \texttt{sscflp.zpl} with a text editor, and fill it with the corresponding ZIMPL statements.
%
See below how the data source can be translated into parameters.
%
ZIMPL uses the intuitive notions of ``sets'' and ``param(eter)s''
to refer to given data of an instance.
%
The naming of the individual parameters is close to the mathematical model~\eqref{eq:model}.


  \lstinputlisting[language=zimpl,
  firstline=1,
  lastline=6,
  caption=Reading input data with ZIMPL]{../data/sscflp.zpl}

The first param is the name of the data file that we want to process.
%
This parameter is used in all subsequent data processing step, with the advantage that the ZIMPL model can be reused for the other data files by changing only this \texttt{datafile} parameter.
%
The next two lines contain the statements to read the number of facilities and customers into ZIMPL parameters \texttt{NJ} and \texttt{NI}, respectively.
%
Here, the command \texttt{use 1} specifies how many lines of the data file should be read in, in this case, only the first line.
%
We use those two parameters to declare two sets \texttt{J} and \texttt{I}, which we use as index sets for remaining data to be read in.

\resizebox{\textwidth}{!}{
  \lstinputlisting[language=zimpl,
  firstline=7,
  lastline=12,
  caption=Reading input data with ZIMPL continued]{../data/sscflp.zpl}
}

We parse the supply of each facility into a new ZIMPL parameter \texttt{s[J]} that is indexed over the set \texttt{J} of facilities.
%
Note that the very first line of the data file does not contain facility information.
%
The \texttt{read} command is executed with the \texttt{skip 1} which tells ZIMPL to skip the first line.
%
Similar \texttt{read} directives are used to declare the opening costs \texttt{f[J]} per facility, the demand \texttt{d[I]} per customer, and the connection costs \texttt{c[J * I]}.
%
Note that this parameter can be conveniently declared over the cross-product of the two individual sets of facilities and customers,
to denote all possible combinations of facilities and customers.
%

The second part of \texttt{sscflp.zpl} uses the data to build the corresponding instance of the mathematical model~\eqref{eq:model}.
%

\resizebox{\textwidth}{!}{
  \lstinputlisting[language=zimpl,
  firstline=13,
  caption=Model Declaration in sscflp.zpl]{../data/sscflp.zpl}
}
%
First, binary variables are declared, which encode the decision which facility should be opened (\texttt{y[J]}),
and the decisions \texttt{x[J*I]} to connect customer and facility, the latter of which are again indexed by the cross-product \texttt{J*I}.
%
The next step is to express the objective function in terms of the variables and parameters.
%
In our case, we declare that the solver should minimize the objective function named ``cost'', and express the connection and opening costs by two summations over the ground sets of facilities and the product of facilities and customers.
%
Finally, the two types of constraints of~\eqref{eq:model} are declared with the help of the \texttt{subto} key word of ZIMPL, followed by the prefixes \texttt{assign} and \texttt{capacity}, respectively.
%
Both declarations use a \texttt{forall} statement, which will cause ZIMPL to create one constraint of type \texttt{assign} per customer in \texttt{I} and one constraint of type \texttt{capacity} per facility in \texttt{J}.

The model \texttt{sscflp.zpl} can be passed directly to ZIMPL to create an instance \texttt{sscflp.lp} via the following command line call.
This invocation reads the data file ``cap61.dat'' as specified in
the first line of \texttt{sscflp.zpl}.

\begin{lstlisting}[language=bash,caption=ZIMPL command line invokation]
zimpl sscflp.zpl

Instructions evaluated: 41042
Name: sscflp.zpl   Variables: 816   Constraints: 66   Non Zeros: 1616
Writing [sscflp.lp]
Writing [sscflp.tbl]
\end{lstlisting}

The data file lists 16 facilities and 50 customers, which is matched by the number of variables and constraints of the resulting model.
%

As promised, we also show how to switch the data set to a different data file.
%
It makes sense to also change the basename of the resulting files to avoid overriding the existing instance \texttt{sscflp.lp} from the previous call.

\begin{lstlisting}[language=bash,caption=ZIMPL command line invokation with options]
zimpl -o capa1 -Ddatafile=capa1.dat sscflp.zpl

Instructions evaluated: 4922017
Name: sscflp.zpl   Variables: 100100   Constraints: 1100   Non Zeros: 200100
Writing [capa1.lp]
Writing [capa1.tbl]
\end{lstlisting}

This call creates a much larger instance \texttt{capa1.lp} based on the data in \texttt{capa1.dat} (100 facilities, 1000 customers).


\paragraph{Remarks.}

For this tutorial, we focus on the most basic instructions of ZIMPL for reading data and constructing mathematical optimization models.
Please refer to the reference manual of ZIMPL\footnote{accessible online via \url{https://zimpl.zib.de/download/zimpl.pdf}} for a complete overview of key words, operators, and further examples of the ZIMPL modeling language.
The examples of the reference manual are contained in the \texttt{examples} directory of the ZIMPL source code.


\section{Solution of the Model with SCIP}

\todo[inline]{Gregor}

\section{Outlook}

\todo[inline]{Thorsten}

\cite{*}

\section*{Bibliography}

\bibliographystyle{alpha}
\bibliography{tutorial}

\end{document}
